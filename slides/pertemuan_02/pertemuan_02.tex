\documentclass[aspectratio=169, table]{beamer}

%\usepackage[beamertheme=./praditatheme]{Pradita}
\usepackage[utf8]{inputenc}
\usepackage{xcolor} % for color
\usepackage{colortbl} % for table color
\usepackage{listings}

% Define Java language style for listings
\lstdefinestyle{JavaStyle}{
	language=Java,
	basicstyle=\ttfamily\tiny,
	keywordstyle=\color{blue},
	commentstyle=\color{gray},
	stringstyle=\color{red},
	breaklines=true,
	showstringspaces=false,
	tabsize=2,
	captionpos=b,
	numbers=left,
	numberstyle=\tiny\color{gray},
	frame=lines,
	backgroundcolor=\color{lightgray!10},
	comment=[l]{//},
	morecomment=[s]{/*}{*/},
	commentstyle=\color{gray}\ttfamily,
	string=[s]{'}{'},
	morestring=[s]{"}{"},
	%	stringstyle=\color{teal}\ttfamily,
	%	showstringspaces=false
}

\lstdefinestyle{sql}{
	language=sql,
	keywords={use, insert, into, values, select, from,
		update, set, delete, create, where, join, left, right, inner, order, by, primary, key},
	ndkeywords={max, min, varchar, int},
	ndkeywordstyle=\color{purple}\bfseries,
	basicstyle=\ttfamily\scriptsize,
	keywordstyle=\color{blue},
	commentstyle=\color{gray},
	stringstyle=\color{red},
	breaklines=true,
	showstringspaces=false,
	tabsize=2,
	captionpos=b,
	numbers=left,
	numberstyle=\tiny\color{gray},
	frame=lines,
	backgroundcolor=\color{lightgray!10},
	comment=[l]{\#},
	morecomment=[s]{/*}{*/},
	commentstyle=\color{gray}\ttfamily,
	string=[s]{'}{'},
	morestring=[s]{"}{"},
	%	stringstyle=\color{teal}\ttfamily,
	%	showstringspaces=false
}

\lstdefinelanguage{bash} {
	keywords={},
	basicstyle=\ttfamily\scriptsize,
	keywordstyle=\color{blue}\bfseries,
	ndkeywords={iex},
	ndkeywordstyle=\color{purple}\bfseries,
	sensitive=true,
	commentstyle=\color{gray},
	stringstyle=\color{red},
	numbers=left,
	numberstyle=\tiny\color{gray},
	breaklines=true,
	frame=lines,
	backgroundcolor=\color{lightgray!10},
	tabsize=2,
	comment=[l]{\#},
	morecomment=[s]{/*}{*/},
	commentstyle=\color{gray}\ttfamily,
	stringstyle=\color{purple}\ttfamily,
	showstringspaces=false
}

\lstdefinestyle{XmlStyle} {
	language=xml,
	keywords={},
	basicstyle=\ttfamily\scriptsize,
	keywordstyle=\color{blue}\bfseries,
	ndkeywords={iex},
	ndkeywordstyle=\color{purple}\bfseries,
	sensitive=true,
	commentstyle=\color{gray},
	stringstyle=\color{red},
	numbers=left,
	numberstyle=\tiny\color{gray},
	breaklines=true,
	frame=lines,
	backgroundcolor=\color{lightgray!10},
	tabsize=2,
	comment=[l]{\#},
	morecomment=[s]{<!--}{-->},
	commentstyle=\color{gray}\ttfamily,
	stringstyle=\color{purple}\ttfamily,
	showstringspaces=false
}

\usetheme{Pradita}

\subtitle{IF220303 - Object-oriented Programming}

\title{\LARGE Build Tools and\\Dependency Manager\vspace{10pt}}
\date[Serial]{\scriptsize {PRU/SPMI/FR-BM-18/0222}}
\author[Pradita]{\small {\textbf{Alfa Yohannis}}}

\begin{document}
	
	\frame{\titlepage}
	

\section{Introduction}
\begin{frame}
	\frametitle{Introduction}
	\begin{itemize}
		\item In software development, the processes of compilation, testing, and dependency management are critical tasks that must be efficiently managed.
		\item Build tools and dependency managers play an important role in automating these tasks, enabling developers to build, manage, and distribute software more easily.
		\item Some common tools used in the Java ecosystem include Ant, Maven, and Gradle.
	\end{itemize}
\end{frame}

	
\section{Compiling Java Code Manually and Creating a JAR}

\begin{frame}
	\frametitle{Manual Java Compilation and JAR Creation}
	\begin{itemize}
		\item Before using build tools like Apache Ant, developers can manually compile Java code and create a JAR file directly through the terminal or command prompt.
		\item Below are the steps to perform this process.
	\end{itemize}
\end{frame}

\subsection{Writing Java Source Code}

\begin{frame}[fragile]
	\vspace{20pt}
	\frametitle{Writing Java Source Code}
	\begin{itemize}
		\item Create a file \texttt{Main.java} in the directory \texttt{src/com/example/} with the following content:
	\end{itemize}
	\begin{lstlisting}[style=JavaStyle]
		package com.example;
		
		public class Main {
			public static void main(String[] args) {
				System.out.println("Hello, World!");
				Util.sayHello("Java Developer");
			}
		}
	\end{lstlisting}
	\begin{itemize}
		\item Additionally, create a file \texttt{Util.java} in the same directory to provide additional functionality:
	\end{itemize}
	\begin{lstlisting}[style=JavaStyle]
		package com.example;
		
		public class Util {
			public static void sayHello(String name) {
				System.out.println("Hello, " + name + "!");
			}
		}
	\end{lstlisting}
\end{frame}

\subsection{Manual Compilation}

\begin{frame}[fragile]
	\frametitle{Manual Compilation}
	\begin{itemize}
		\item Use the following command to manually compile the Java code without using a build tool:
	\end{itemize}
	\begin{lstlisting}[language=bash]
		mkdir -p bin
		javac -d bin src/com/example/*.java
	\end{lstlisting}
	\begin{itemize}
		\item This command will compile all \texttt{.java} files in the \texttt{src/com/example/} directory and save them in the \texttt{bin/} directory.
	\end{itemize}
\end{frame}

\subsection{Running the Program Manually}

\begin{frame}[fragile]
	\frametitle{Running the Program Manually}
	\begin{itemize}
		\item After successful compilation, run the program with the following command:
	\end{itemize}
	\begin{lstlisting}[language=bash]
		java -cp bin com.example.Main
	\end{lstlisting}
	\begin{itemize}
		\item The output will be:
	\end{itemize}
	\begin{lstlisting}[language=bash]
		Hello, World!
		Hello, Java Developer!
	\end{lstlisting}
\end{frame}

\subsection{Creating the JAR File}

\begin{frame}[fragile]
	\frametitle{Creating the JAR File}
	\begin{itemize}
		\item To create a JAR file from the compiled code, use the following command:
	\end{itemize}
	\begin{lstlisting}[language=bash]
		jar cf bin/app.jar -C bin .
	\end{lstlisting}
	\begin{itemize}
		\item This command will create a file \texttt{app.jar} in the \texttt{bin/} directory.
	\end{itemize}
\end{frame}

\subsection{Compiling a New Program Using a JAR File}

\begin{frame}
	\frametitle{Compiling a New Program Using a JAR File}
	\begin{itemize}
		\item After creating \texttt{app.jar}, it can be used in other projects or new programs that require functions from \texttt{Util.java}.
	\end{itemize}
\end{frame}

\subsubsection{Writing Java Source Code}

\begin{frame}[fragile]
	\frametitle{Writing Java Source Code}
	\begin{itemize}
		\item Create a new file \texttt{NewMain.java} in the directory \texttt{src/com/example/} with the following content:
	\end{itemize}
	\begin{lstlisting}[style=JavaStyle]
		package com.example;
		
		import com.example.Util;
		
		public class NewMain {
			public static void main(String[] args) {
				System.out.println("This is a new program using app.jar!");
				Util.sayHello("JAR User");
			}
		}
	\end{lstlisting}
	\begin{itemize}
		\item This code reuses the \texttt{sayHello} function from \texttt{Util.java}, which is already included in \texttt{app.jar}.
	\end{itemize}
\end{frame}

\subsubsection{Compiling the New Program with the JAR File}

\begin{frame}[fragile]
	\frametitle{Compiling the New Program with the JAR File}
	\begin{itemize}
		\item Use the following command to compile \texttt{NewMain.java} with the \texttt{app.jar}:
	\end{itemize}
	\begin{lstlisting}[language=bash]
		mkdir -p bin
		javac -cp bin/app.jar -d bin src/com/example/NewMain.java
	\end{lstlisting}
	\begin{itemize}
		\item This command instructs the Java compiler to use \texttt{app.jar} as part of the classpath to access \texttt{Util.java}.
	\end{itemize}
\end{frame}

\subsubsection{Running the New Program with the JAR File}

\begin{frame}[fragile]
	\frametitle{Running the New Program with the JAR File}
	\begin{itemize}
		\item After successful compilation, run the program with the following command:
	\end{itemize}
	\begin{lstlisting}[language=bash]
		java -cp bin:bin/app.jar com.example.NewMain
	\end{lstlisting}
	\begin{itemize}
		\item The output will be:
	\end{itemize}
	\begin{lstlisting}[language=bash]
		This is a new program using app.jar!
		Hello, JAR User!
	\end{lstlisting}
	\begin{itemize}
		\item This demonstrates how we successfully reused the function from \texttt{app.jar} in a new program without rewriting \texttt{Util.java}.
	\end{itemize}
\end{frame}

\subsection{Creating a JAR File for the New Program}

\begin{frame}
	\frametitle{Creating a JAR File for the New Program}
	\begin{itemize}
		\item After compiling \texttt{NewMain.java} and using \texttt{app.jar} as a dependency, the next step is to create a new JAR file that includes the program.
	\end{itemize}
\end{frame}

\subsubsection{Creating the New JAR File}

\begin{frame}[fragile]
	\frametitle{Creating the New JAR File}
	\begin{itemize}
		\item Use the following command to create a new JAR file \texttt{newapp.jar} that includes \texttt{NewMain.class} and still uses \texttt{app.jar} as a dependency:
	\end{itemize}
	\begin{lstlisting}[language=bash]
		jar cf bin/newapp.jar -C bin .
	\end{lstlisting}
	\begin{itemize}
		\item This command will create the file \texttt{newapp.jar} in the \texttt{bin/} directory, containing all the compiled class files.
	\end{itemize}
\end{frame}

\subsubsection{Running the Program from the New JAR}

\begin{frame}[fragile]
	\frametitle{Running the Program from the New JAR}
	\begin{itemize}
		\item Run the program from \texttt{newapp.jar} using the following command:
	\end{itemize}
	\begin{lstlisting}[language=bash]
		java -cp bin/newapp.jar:bin/app.jar com.example.NewMain
	\end{lstlisting}
	\begin{itemize}
		\item If using Windows, use a semicolon (\texttt{;}) as the classpath separator:
	\end{itemize}
	\begin{lstlisting}[language=bash]
		java -cp bin/newapp.jar;bin/app.jar com.example.NewMain
	\end{lstlisting}
\end{frame}

\subsubsection{Running the New JAR with the \texttt{-jar} Option}

\begin{frame}[fragile]
	\frametitle{\LARGE{Running the New JAR with the \texttt{-jar} Option (Part 1)}}
	\begin{itemize}
		\item To run \texttt{newapp.jar} directly with the \texttt{-jar} option, add a \texttt{MANIFEST.MF} file to define the entry point:
		\item First, create the \texttt{META-INF} directory and the manifest file:
	\end{itemize}
	\begin{lstlisting}[language=bash]
		mkdir -p bin/META-INF
		echo "Main-Class: com.example.NewMain" > bin/META-INF/MANIFEST.MF
	\end{lstlisting}
\end{frame}

\begin{frame}[fragile]
	\frametitle{\LARGE{Running the New JAR with the \texttt{-jar} Option (Part 2)}}
	\begin{itemize}
		\item Then, recreate the \texttt{newapp.jar} file including the manifest:
	\end{itemize}
	\begin{lstlisting}[language=bash]
		jar cfm bin/newapp.jar bin/META-INF/MANIFEST.MF -C bin .
	\end{lstlisting}
	\begin{itemize}
		\item Now, run the program with the following command:
	\end{itemize}
	\begin{lstlisting}[language=bash]
		java -jar bin/newapp.jar
	\end{lstlisting}
	\begin{itemize}
		\item If an error occurs due to external dependencies, use the \texttt{-cp} option to include \texttt{app.jar}:
	\end{itemize}
	\begin{lstlisting}[language=bash]
		java -cp bin/newapp.jar:bin/app.jar com.example.NewMain
	\end{lstlisting}
\end{frame}


\subsubsection{Conclusion}

\begin{frame}
	\frametitle{Conclusion}
	\begin{itemize}
		\item Successfully used \texttt{app.jar} in a new program.
		\item Compiled and ran the new program with a classpath containing \texttt{app.jar}.
		\item Created an executable \texttt{newapp.jar}.
		\item Ran \texttt{newapp.jar} using both the \texttt{-cp} and \texttt{-jar} options.
	\end{itemize}
\end{frame}


%%% ANT
\section{Apache Ant}

\begin{frame}
	\frametitle{Apache Ant}
	Apache Ant is one of the oldest build tools used in the Java ecosystem. Ant uses an XML file as a script to define build tasks such as compiling source code, running unit tests, and generating JAR or WAR files.
\end{frame}

\subsection{Main Features of Ant}

\begin{frame}
	\frametitle{Main Features of Ant}
	\begin{itemize}
		\item Uses XML for configuration.
		\item Provides high flexibility through customizable targets and tasks.
		\item Does not have built-in dependency management.
	\end{itemize}
\end{frame}

\subsection{Installing Apache Ant}

\begin{frame}
	\frametitle{Installing Apache Ant}
	To use Apache Ant, follow these steps:
	\begin{enumerate}
		\item Download Apache Ant from the official website: \url{https://ant.apache.org/}.
		\item Extract the downloaded archive to a preferred directory.
		\item Add environment variables:
		\begin{itemize}
			\item Add the \texttt{bin} directory of Ant to the \texttt{PATH} variable.
			\item Set the \texttt{ANT\_HOME} variable to the Ant installation directory.
			\item Ensure that the \texttt{JAVA\_HOME} is configured correctly.
		\end{itemize}
	\end{enumerate}
\end{frame}

\subsubsection{Verifying Installation}

\begin{frame}[fragile]
	\frametitle{Verifying Installation}
	After installation, run the following command in the terminal or command prompt to verify the installation:
	\begin{lstlisting}[language=bash]
		ant -version
	\end{lstlisting}
	\begin{itemize}
		\item If installation is successful, the output will display the installed version of Ant.
	\end{itemize}
\end{frame}

\subsection{Basic Usage of Apache Ant}

\subsubsection{Project Structure}

\begin{frame}[fragile]
	\frametitle{Project Structure}
	Before using Ant, ensure that the project structure is as follows:
	\begin{lstlisting}[language=bash]
		ProjectRoot/
		|-- lib/
		|   |-- app.jar
		|-- src/
		|   |-- com/
		|   |   |-- example/
		|   |   |   |-- Main.java
		|-- build.xml
	\end{lstlisting}
	\begin{itemize}
		\item The \texttt{lib/} directory contains the \texttt{app.jar} file created earlier, which will be used in this project.
	\end{itemize}
\end{frame}

\subsubsection{Source Code \texttt{Main.java}}

\begin{frame}[fragile]
	\frametitle{Source Code \texttt{Main.java}}
	Create the \texttt{Main.java} file inside the \texttt{src/com/example/} directory with the following content:
	\begin{lstlisting}[style=JavaStyle]
		package com.example;
		
		import com.example.Util;
		
		public class Main {
			public static void main(String[] args) {
				System.out.println("Running with Ant and External JAR!");
				Util.sayHello("Ant User");
			}
		}
	\end{lstlisting}
	\begin{itemize}
		\item This code uses the \texttt{Util} class, which has already been compiled and packaged in the \texttt{app.jar}.
	\end{itemize}
\end{frame}

\subsubsection{Writing the \texttt{build.xml} File}

\begin{frame}[fragile]
	\vspace{20pt}
	\frametitle{Writing the \texttt{build.xml} File}
	Create the \texttt{build.xml} file in the root directory of the project with the following content:
	\begin{lstlisting}[style=XmlStyle]
<project name="HelloWorld" default="run">
<property name="src.dir" value="src" />
<property name="bin.dir" value="bin" />
<property name="lib.dir" value="lib" />
<property name="jarfile" value="${lib.dir}/app.jar" />

<target name="init"><mkdir dir="${bin.dir}" /></target>

<target name="compile" depends="init">
<javac srcdir="${src.dir}" destdir="${bin.dir}">
<classpath>
<fileset dir="${lib.dir}" includes="**/*.jar" />
<!-- Add additional classpaths if needed -->
</classpath>
</javac>
</target>

<target name="run" depends="compile">
<java classname="com.example.Main" classpath="${bin.dir}:${lib.dir}/app.jar" fork="true" />
</target>
</project>
	\end{lstlisting}
\end{frame}

\subsubsection{Running the Build with Ant}

\begin{frame}[fragile]
	\frametitle{Running the Build with Ant}
	Run the following commands to execute tasks in the \texttt{build.xml}:
	\begin{itemize}
		\item To compile the code with the external JAR:
	\end{itemize}
	\begin{lstlisting}[language=bash]
		ant compile
	\end{lstlisting}
	\begin{itemize}
		\item To run the application:
	\end{itemize}
	\begin{lstlisting}[language=bash]
		ant run
	\end{lstlisting}
	\begin{itemize}
		\item With this configuration, Ant will compile the Java code with the \textbf{external JAR} and then run the program using \textbf{functions from that JAR}.
	\end{itemize}
\end{frame}


%%%% -- MAVEN

\section{Apache Maven}
\begin{frame}
	\frametitle{Apache Maven}
	Apache Maven is a build tool and dependency manager commonly used in the Java ecosystem. By using the \textbf{Project Object Model (POM)} concept, Maven simplifies project management, including compilation, testing, and creating artifacts such as JAR and WAR files.
\end{frame}

\subsection{Key Features of Maven}
\begin{frame}
	\frametitle{Key Features of Maven}
	\begin{itemize}
		\item Uses the \texttt{pom.xml} file for project configuration.
		\item Provides a standard build lifecycle (\texttt{compile}, \texttt{test}, \texttt{package}, \texttt{install}, etc.).
		\item Has an integrated dependency management system.
	\end{itemize}
\end{frame}

\subsubsection{Download and Installation of Maven}
\begin{frame}
	\frametitle{Download and Installation of Maven}
	\begin{enumerate}
		\item Download Apache Maven from the official website: \url{https://maven.apache.org/download.cgi}
		\item Extract the downloaded archive to your preferred directory.
		\item Add environment variables:
		\begin{itemize}
			\item Add the \texttt{bin} directory of Maven to the \texttt{PATH} variable.
			\item Set the \texttt{MAVEN\_HOME} variable to the Maven installation directory.
			\item Ensure the \texttt{JAVA\_HOME} variable is correctly configured.
		\end{itemize}
	\end{enumerate}
\end{frame}

\subsubsection{Verifying Installation}
\begin{frame}[fragile]
	\frametitle{Verifying Installation}
	After the installation is complete, run the following command in the terminal or command prompt to verify the installation:
	\begin{lstlisting}[language=bash]
		mvn -version
	\end{lstlisting}
	If the installation is successful, the output will display the installed version of Maven.
\end{frame}

\begin{frame}[fragile]
	\vspace{20pt}
	\frametitle{Creating a Maven Project}
	To create a new project using Maven, follow these steps:
	\begin{enumerate}
		\item Open the terminal or command prompt.
		\item Use the following Maven command to create a new Maven project:
		\begin{lstlisting}[language=bash]
			mvn archetype:generate -DgroupId=com.example -DartifactId=my-app -DarchetypeArtifactId=maven-archetype-quickstart -DinteractiveMode=false
		\end{lstlisting}
		The above command will create a Maven project with \texttt{groupId} as \texttt{com.example} and \texttt{artifactId} as \texttt{my-app}.
		\item After running the command, Maven will download the project template and create the required directory structure.
		\item Navigate to the newly created project directory:
		\begin{lstlisting}[language=bash]
			cd my-app
		\end{lstlisting}
	\end{enumerate}
\end{frame}

\subsection{Maven Project Structure}
\begin{frame}[fragile]
	\frametitle{Maven Project Structure}
	Before using Maven, ensure the project structure is as follows:
	\begin{lstlisting}[language=bash]
		ProjectRoot/
		|-- src/
		|   |-- main/
		|   |   |-- java/
		|   |   |   |-- com/
		|   |   |   |   |-- example/
		|   |   |   |   |   |-- Main.java
		|   |-- test/
		|       |-- java/
		|-- pom.xml
	\end{lstlisting}
\end{frame}

\subsubsection{Writing the \texttt{pom.xml} File}
\begin{frame}[fragile]
	\vspace{20pt}
	\frametitle{Writing the \texttt{pom.xml} File}
	The \texttt{pom.xml} file is the main configuration in a Maven project.
	\begin{lstlisting}[style=XmlStyle]
		<project xmlns="http://maven.apache.org/POM/4.0.0"
		xmlns:xsi="http://www.w3.org/2001/XMLSchema-instance"
		xsi:schemaLocation="http://maven.apache.org/POM/4.0.0
		http://maven.apache.org/xsd/maven-4.0.0.xsd">
		<modelVersion>4.0.0</modelVersion>
		
		<groupId>com.example</groupId>
		<artifactId>helloworld</artifactId>
		<version>1.0-SNAPSHOT</version>
		
		<dependencies>
		<!-- Using Google Guava from Maven Central -->
		<dependency>
		<groupId>com.google.guava</groupId>
		<artifactId>guava</artifactId>
		<version>33.4.0-jre</version>
		</dependency>
		</dependencies>
		</project>
	\end{lstlisting}
\end{frame}

\subsubsection{Source Code for \texttt{Main.java}}
\begin{frame}[fragile]
	\frametitle{Source Code for \texttt{Main.java}}
	Create the \texttt{Main.java} file in the directory \texttt{src/main/java/com/example/} with the following contents:
	\begin{lstlisting}[style=JavaStyle]
		package com.example;
		
		import com.google.common.base.Joiner;
		
		public class Main {
			public static void main(String[] args) {
				System.out.println("Hello, Maven!");
				
				// Using a function from the Guava library (downloaded from Maven Central)
				String result = Joiner.on(", ").join("Apple", "Banana", "Cherry");
				System.out.println("Joined String: " + result);
			}
		}
	\end{lstlisting}
\end{frame}

\subsubsection{Creating a JAR File with Maven}
\begin{frame}[fragile]
	\frametitle{Creating a JAR File with Maven}
	To compile the code and create a JAR file, run the following command:
	\begin{lstlisting}[language=bash]
		mvn clean package
	\end{lstlisting}
	The output will be in the \texttt{target/} directory:
	\begin{lstlisting}[language=bash]
		target/
		|-- helloworld-1.0-SNAPSHOT.jar
	\end{lstlisting}
\end{frame}

\subsubsection{Running the Program from the JAR}
\begin{frame}[fragile]
	\frametitle{Running the Program from the JAR}
	Use the following command to run the program from the generated JAR:
	\begin{lstlisting}[language=bash]
		java -cp target/helloworld-1.0-SNAPSHOT.jar:~/.m2/repository/com/google/guava/guava/31.1-jre/guava-31.1-jre.jar com.example.Main
	\end{lstlisting}
\end{frame}

\subsubsection{Running with \texttt{-jar}}
\begin{frame}[fragile]
	\vspace{20pt}
	\frametitle{Running with \texttt{-jar}}
	If you want to create a JAR that can be directly run with the \texttt{-jar} option, add the following plugin configuration to your \texttt{pom.xml}.
\end{frame}

\begin{frame}[fragile]
	\vspace{20pt}
	\frametitle{Maven Assembly Plugin Configuration (Part 1)}
	First, add the Maven Assembly Plugin to create a fat JAR that includes all dependencies. This ensures the resulting JAR can be run with the \texttt{-jar} option.
	\begin{lstlisting}[style=XmlStyle]
		<plugin>
		<groupId>org.apache.maven.plugins</groupId>
		<artifactId>maven-assembly-plugin</artifactId>
		<version>3.3.0</version>
		<executions>
		<execution>
		<phase>package</phase>
		<goals>
		<goal>single</goal>
		</goals>
		<configuration>
		<descriptorRefs>
		<descriptorRef>jar-with-dependencies</descriptorRef>
		</descriptorRefs>
	\end{lstlisting}
\end{frame}

\begin{frame}[fragile]
	\frametitle{Maven Assembly Plugin Configuration (Part 2)}
	Continue by adding the rest of the configuration for the Maven Assembly Plugin in your \texttt{pom.xml} file. This includes specifying the main class in the manifest.
	\begin{lstlisting}[style=XmlStyle]
		<!-- Specify the Main-Class in the manifest -->
		<archive>
		<manifestEntries>
		<Main-Class>com.example.Main</Main-Class>
		</manifestEntries>
		</archive>
		</configuration>
		</execution>
		</executions>
		</plugin>
	\end{lstlisting}
\end{frame}


\begin{frame}[fragile]
	\frametitle{Maven Jar Plugin Configuration}
	\vspace{20pt}
	Next, add the Maven Jar Plugin to specify the main class that should be executed when the JAR is run with the \texttt{-jar} option.
	\begin{lstlisting}[style=XmlStyle]
		<plugin>
		<groupId>org.apache.maven.plugins</groupId>
		<artifactId>maven-jar-plugin</artifactId>
		<version>3.2.0</version>
		<configuration>
		<archive>
		<manifestEntries>
		<!-- Specify the main class to be executed -->
		<Main-Class>com.example.Main</Main-Class>
		</manifestEntries>
		</archive>
		</configuration>
		</plugin>
	\end{lstlisting}
\end{frame}


\subsubsection{Build JAR and Run It}
\begin{frame}[fragile]
	\frametitle{Build JAR and Run It}
	Then, run the following command:
	\begin{lstlisting}[language=bash]
		mvn clean package
	\end{lstlisting}
	After that, the program can be run directly with:
	\begin{lstlisting}[language=bash]
		java -jar target/helloworld-1.0-SNAPSHOT.jar
	\end{lstlisting}
\end{frame}

\subsubsection{Running with \texttt{mvn exec:java}}
\begin{frame}[fragile]
	\frametitle{Running with \texttt{mvn exec:java}}
	We can also run the program directly using Maven without creating the JAR first.
	Use the following command to run the application:
	\begin{lstlisting}[language=bash]
		mvn clean compile exec:java -Dexec.mainClass="com.example.Main"
	\end{lstlisting}
	This command will:
	\begin{itemize}
		\item Clean up any previous builds (\texttt{clean}).
		\item Compile the source code (\texttt{compile}).
		\item Run the \texttt{Main} class using the \texttt{exec-maven-plugin}.
	\end{itemize}
\end{frame}

\subsubsection{Running with \texttt{mvn package} and \texttt{mvn exec:java}}
\begin{frame}[fragile]
	\frametitle{\LARGE{Running with \texttt{mvn package} and \texttt{mvn exec:java}}}
	If you want to ensure the code is compiled before running, use the following commands:
	\begin{lstlisting}[language=bash]
		mvn clean package
		mvn exec:java -Dexec.mainClass="com.example.Main"
	\end{lstlisting}
\end{frame}

\subsubsection{Conclusion}
\begin{frame}
	\frametitle{Conclusion}
	With these steps, we have successfully:
	\begin{itemize}
		\item Installed and verified Apache Maven.
		\item Used external dependencies from Maven Central (Google Guava).
		\item Managed dependencies without manually downloading and adding JAR files.
		\item Compiled the code and generated an executable JAR file.
		\item Run the program with dependencies from Maven.
		\item Run the program directly using Maven without first creating a JAR.
	\end{itemize}
\end{frame}

%%% GRADLE
\section{Gradle}
\begin{frame}[fragile]
	\frametitle{Gradle}
	Gradle is a modern, flexible, and fast build tool used for managing projects in Java, Kotlin, Groovy, and other languages. It offers a more concise declarative approach compared to Ant and Maven, supporting Groovy or Kotlin-based build scripts.
\end{frame}

\subsection{Key Features of Gradle}
\begin{frame}[fragile]
	\frametitle{Key Features of Gradle}
	\begin{itemize}
		\item Uses \texttt{build.gradle} or \texttt{build.gradle.kts} files for project configuration.
		\item Has an integrated dependency management system, supporting Maven Central, JCenter, and local repositories.
		\item Supports incremental builds to improve compilation speed.
		\item Uses declarative tasks which are more concise compared to Ant and Maven.
	\end{itemize}
\end{frame}

\subsubsection{Gralde Download and Installation}
\begin{frame}[fragile]
	\frametitle{Gradle Download and Installation}
	\begin{enumerate}
		\item Download Gradle from the official website: \url{https://gradle.org/install/}
		\item Extract the downloaded archive to your preferred directory.
		\item Add environment variables:
		\begin{itemize}
			\item Add the \texttt{bin} directory of Gradle to the \texttt{PATH} variable.
			\item Set the \texttt{GRADLE\_HOME} variable to the Gradle installation directory.
			\item Ensure that \texttt{JAVA\_HOME} is correctly configured.
		\end{itemize}
	\end{enumerate}
\end{frame}

\subsubsection{Verify Installation}
\begin{frame}[fragile]
	\frametitle{Verify Installation}
	After installation, run the following command in your terminal or command prompt to verify the installation:
	\begin{lstlisting}[language=bash]
		gradle -version
	\end{lstlisting}
	If the installation is successful, the output will display the installed version of Gradle.
\end{frame}

\begin{frame}[fragile]
	\vspace{20pt}
	\frametitle{Creating a Gradle Project}
	To create a new project using Gradle, follow these steps:
	\begin{enumerate}
		\item Open the terminal or command prompt. Create a directory which is project folder:
		\begin{lstlisting}[language=bash]
			mkdir <project-name>
		\end{lstlisting}
		\item Navigate to the newly created project directory:
		\begin{lstlisting}[language=bash]
			cd <project-name>
		\end{lstlisting}
		\item Use the following Gradle command to create a new Gradle project:
		\begin{lstlisting}[language=bash]
			gradle init --type java-application
		\end{lstlisting}
		The above command will generate a new Java application project with the default structure. After running the command, Gradle will create the necessary files and directories.
		
	\end{enumerate}
\end{frame}

\subsection{Basic Usage of Gradle}
\begin{frame}[fragile]
	\frametitle{Basic Usage of Gradle}
	\subsubsection{Project Structure}
	Before using Gradle, ensure your project has the following structure:
	\begin{lstlisting}[language=bash]
		ProjectRoot/
		|-- src/
		|   |-- main/
		|   |   |-- java/
		|   |   |   |-- com/
		|   |   |   |   |-- example/
		|   |   |   |   |   |-- Main.java
		|   |-- test/
		|       |-- java/
		|-- build.gradle
	\end{lstlisting}
\end{frame}

\subsubsection{Writing the \texttt{build.gradle} File}
\begin{frame}[fragile]
	\frametitle{Writing the \texttt{build.gradle} File - Part 1}
	The \texttt{build.gradle} file is the main configuration file for Gradle projects. Here’s an example with the basic setup for the Gradle project:
	\begin{lstlisting}[style=XmlStyle]
		plugins {
			id 'java'
		}
		
		group 'com.example'
		version '1.0-SNAPSHOT'
		
		repositories {
			mavenCentral()
		}
	\end{lstlisting}
\end{frame}

\begin{frame}[fragile]
	\vspace{20pt}
	\frametitle{Writing the \texttt{build.gradle} File - Part 2}
	Continuing from the previous configuration, here’s the part where dependencies are added, including a local JAR file (\texttt{app.jar}) and an external dependency (Google Guava from Maven Central):
	\begin{lstlisting}[style=XmlStyle]
		dependencies {
			// Using Google Guava from Maven Central
			implementation 'com.google.guava:guava:31.1-jre'

		}
		
		tasks.named('jar') {
			manifest {
				attributes 'Main-Class': 'com.example.Main'
			}
		}
	\end{lstlisting}
\end{frame}

\subsubsection{Source Code for \texttt{Main.java}}
\begin{frame}[fragile]
	\frametitle{Source Code for \texttt{Main.java}}
	Create the \texttt{Main.java} file inside \texttt{src/main/java/com/example/} with the following content:
	\begin{lstlisting}[style=JavaStyle]
		package com.example;
		
		import com.google.common.base.Joiner;
		
		public class Main {
			public static void main(String[] args) {
				System.out.println("Hello, Gradle!");
				
				// Using function from Google Guava (Maven Central)
				String result = Joiner.on(", ").join("Apple", "Banana", "Cherry");
				System.out.println("Joined String: " + result);
				
			}
		}
	\end{lstlisting}
\end{frame}

\subsubsection{Creating a JAR File with Gradle}
\begin{frame}[fragile]
	\frametitle{Creating a JAR File with Gradle}
	To compile the code and create a JAR file, run the following command:
	\begin{lstlisting}[language=bash]
		gradle clean build
	\end{lstlisting}
	The output will be located in the \texttt{build/libs/} directory:
	\begin{lstlisting}[language=bash]
		build/libs/
		|-- helloworld-1.0-SNAPSHOT.jar
	\end{lstlisting}
\end{frame}

\subsubsection{Running the Program from the JAR}
\begin{frame}[fragile]
	\vspace{20pt}
	\frametitle{Running the Program from the JAR}
	Use the following command to run the program from the generated JAR:
	\begin{lstlisting}[language=bash]
		java -cp build/libs/helloworld-1.0-SNAPSHOT.jar:lib/app.jar:~/.gradle/caches/modules-2/files-2.1/com.google.guava/guava/31.1-jre com.example.Main
	\end{lstlisting}
	To make the JAR file executable with the \texttt{-jar} option, add the following configuration to your \texttt{build.gradle}:
	\begin{lstlisting}[style=XmlStyle]
		jar {
			manifest {
				attributes 'Main-Class': 'com.example.Main'
			}
			from {
				configurations.runtimeClasspath.collect { it.isDirectory() ? it : zipTree(it) }
			}
		}
	\end{lstlisting}
\end{frame}

\subsubsection{Running the Program Using Gradle}
\begin{frame}[fragile]
	\vspace{20pt}
	\frametitle{Running the Program Using Gradle}
	In addition to running the program with the \texttt{java -cp} command, you can also run the application directly with Gradle.\\
	\textbf{Running with \texttt{gradle run}}
	Use the following command to run the application without creating the JAR first:
	\begin{lstlisting}[language=bash]
		gradle run
	\end{lstlisting}
	If the \texttt{run} task is not available, add the following plugin to your \texttt{build.gradle}:
	\begin{lstlisting}[style=XmlStyle]
		plugins {
			id 'application'
		}
		
		application {
			mainClass = 'com.example.Main'
		}
	\end{lstlisting}
\end{frame}

\subsubsection{Conclusion}
\begin{frame}[fragile]
	\frametitle{Conclusion}
	With these steps, we have successfully:
	\begin{itemize}
		\item Installed and verified Gradle.
		\item Used external dependencies from Maven Central (Google Guava).
		\item Used a local JAR file (\texttt{app.jar}) in the Gradle project.
		\item Compiled the code and generated a runnable JAR file.
		\item Run the program with Gradle dependencies.
		\item Run the program directly with Gradle without creating a JAR first.
	\end{itemize}
\end{frame}

\section{Comparison of Ant, Maven, and Gradle}
\begin{frame}[fragile]
	\frametitle{Comparison of Build Tools}
	Each build tool has its own characteristics, with advantages and disadvantages that need to be considered based on the project's needs. Below is a comparison of Apache Ant, Apache Maven, and Gradle.
\end{frame}


\subsection{Apache Ant}

\begin{frame}[fragile]
	\frametitle{Advantages of Ant}
	\textbf{Ant} is an XML-based build tool designed to automate the build process. Ant is more flexible than traditional Makefiles, but it lacks an integrated dependency management system.
	\begin{itemize}
		\item Highly flexible and can be used for various custom build tasks.
		\item Does not enforce a specific project structure.
		\item Supports extensions through additional libraries.
	\end{itemize}
\end{frame}

\begin{frame}[fragile]
	\frametitle{Disadvantages of Ant}
	\begin{itemize}
		\item Lacks built-in dependency management, requiring manual JAR management or use of Ivy.
		\item Uses XML for configuration, which is often more verbose and harder to read compared to DSLs like Groovy or Kotlin.
		\item Does not have a structured build lifecycle like Maven and Gradle, so developers need to define all steps explicitly.
	\end{itemize}
\end{frame}

\subsection{Apache Maven}
\begin{frame}[fragile]
	\frametitle{Advantages of Maven}
	\textbf{Maven} is a build tool based on the \textbf{Project Object Model (POM)} concept and has built-in dependency management. Maven uses XML for configuration and provides a clear build lifecycle.
	\begin{itemize}
		\item Has an automatic dependency management system with support for Maven Central repository.
		\item Standard build lifecycle, including \texttt{compile}, \texttt{test}, \texttt{package}, and \texttt{install}.
		\item More consistent and conventional project structure, making collaboration easier.
	\end{itemize}
\end{frame}

\begin{frame}[fragile]
	\frametitle{Disadvantages of Maven}
	\begin{itemize}
		\item XML-based configuration can become long and hard to read.
		\item The build lifecycle can feel rigid if the project requires a lot of customization.
		\item Performance is slower compared to Gradle due to the lack of efficient incremental build support.
	\end{itemize}
\end{frame}

\subsection{Gradle}

\begin{frame}[fragile]
	\textbf{Gradle} is a modern build tool designed to improve efficiency and flexibility in building software projects. Gradle uses a declarative language based on Groovy or Kotlin DSL for configuration.
	\frametitle{Advantages of Gradle}
	\begin{itemize}
		\item Uses Groovy/Kotlin-based DSL, which is more concise than XML in Ant and Maven.
		\item Supports incremental builds, making it faster than Maven.
		\item High flexibility for customization, allowing developers to easily tailor the build process.
		\item Strong dependency management, similar to Maven but with simpler syntax.
	\end{itemize}
\end{frame}

\begin{frame}[fragile]
	\frametitle{Disadvantages of Gradle}
	\begin{itemize}
		\item Higher learning curve due to the use of Groovy/Kotlin DSL.
		\item Documentation is not as comprehensive as Maven, especially for more complex features.
		\item Not all Maven plugins are available for Gradle, so migrating from Maven may require extra effort.
	\end{itemize}
\end{frame}

\subsection{Comparison Table}

\begin{frame}[fragile]
	\frametitle{Comparison Table}
	The table below summarizes the main differences between Ant, Maven, and Gradle:
	
	\begin{table}[h]
		\centering
		\scriptsize
		% Set line thickness and color
		\setlength{\arrayrulewidth}{0.01mm}  % Line thickness
		\arrayrulecolor[HTML]{000000} % Line color (orange in this case)
		
		\begin{tabular}{p{0.2\textwidth} p{0.22\textwidth} p{0.22\textwidth} p{0.22\textwidth}}
			\hline
			\textbf{Feature} & \textbf{Ant} & \textbf{Maven} & \textbf{Gradle} \\
			\hline
			Dependency Management & None (requires Ivy) & Built-in (Maven Central) & Built-in (Maven Central, Gradle Plugin) \\
			\hline
			Project Structure & Flexible & Conventional & Conventional \\
			\hline
			Configuration Format & XML & XML & Groovy/Kotlin DSL \\
			\hline
			Build Lifecycle & None (manual) & Present & Present \\
			\hline
			Build Speed & Slow & Moderate & Fast (Incremental Build) \\
			\hline
			Flexibility & High & Moderate & High \\
			\hline
			Learning Curve & Low & Moderate & High \\
			\hline
		\end{tabular}
		\caption{Comparison of Ant, Maven, and Gradle}
		\label{tab:comparison}
	\end{table}
\end{frame}




\subsection{Conclusion}
\begin{frame}[fragile]
	\frametitle{Conclusion}
	The choice of build tool depends on the project's needs:
	\begin{itemize}
		\item \textbf{Use Ant} if full control over the build process is required and built-in dependency management is not needed.
		\item \textbf{Use Maven} if you want a clear project structure, automatic build lifecycle, and strong dependency management.
		\item \textbf{Use Gradle} if you need faster builds, flexibility, and more concise configuration compared to Maven.
	\end{itemize}
	Gradle is increasingly popular for modern projects due to its speed and customization capabilities. However, Maven remains a solid choice for enterprise projects that prioritize stability and mature documentation. Ant is better suited for legacy projects or highly customized build systems.
\end{frame}

\end{document}
