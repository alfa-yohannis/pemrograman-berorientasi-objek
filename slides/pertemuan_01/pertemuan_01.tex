\documentclass[aspectratio=169, table]{beamer}

%\usepackage[beamertheme=./praditatheme]{Pradita}
\usepackage[utf8]{inputenc}

\usepackage{listings}

% Define Java language style for listings
\lstdefinestyle{JavaStyle}{
	language=Java,
	basicstyle=\ttfamily\tiny,
	keywordstyle=\color{blue},
	commentstyle=\color{gray},
	stringstyle=\color{red},
	breaklines=true,
	showstringspaces=false,
	tabsize=2,
	captionpos=b,
	numbers=left,
	numberstyle=\tiny\color{gray},
	frame=lines,
	backgroundcolor=\color{lightgray!10},
	comment=[l]{//},
	morecomment=[s]{/*}{*/},
	commentstyle=\color{gray}\ttfamily,
	string=[s]{'}{'},
	morestring=[s]{"}{"},
	%	stringstyle=\color{teal}\ttfamily,
	%	showstringspaces=false
}

\lstdefinestyle{sql}{
	language=sql,
	keywords={use, insert, into, values, select, from,
		update, set, delete, create, where, join, left, right, inner, order, by, primary, key},
	ndkeywords={max, min, varchar, int},
	ndkeywordstyle=\color{purple}\bfseries,
	basicstyle=\ttfamily\footnotesize,
	keywordstyle=\color{blue},
	commentstyle=\color{gray},
	stringstyle=\color{red},
	breaklines=true,
	showstringspaces=false,
	tabsize=2,
	captionpos=b,
	numbers=left,
	numberstyle=\tiny\color{gray},
	frame=lines,
	backgroundcolor=\color{lightgray!10},
	comment=[l]{\#},
	morecomment=[s]{/*}{*/},
	commentstyle=\color{gray}\ttfamily,
	string=[s]{'}{'},
	morestring=[s]{"}{"},
	%	stringstyle=\color{teal}\ttfamily,
	%	showstringspaces=false
}


\usetheme{Pradita}

\subtitle{IF220303 - Object-oriented Programming}

\title{\LARGE Foundational Concepts of\\Object-oriented Programming \vspace{10pt}}
\date[Serial]{\scriptsize {PRU/SPMI/FR-BM-18/0222}}
\author[Pradita]{\small {\textbf{Alfa Yohannis}}}

\begin{document}
	
	\frame{\titlepage}
%	
%	
%	{
%		\setbeamertemplate{navigation symbols}{}
%		\setbeamertemplate{footline}{}		
%		\begin{frame}
%			\frametitle{TOGAF Architecture Development Method (ADM)}
%			\framesubtitle{\hspace{1cm}}
%			\vspace{10pt}
%			\begin{center}
%%				\includegraphics[width=0.38\textwidth]{../figures/adm}
%			\end{center}
%		\end{frame}
%	}
	
\begin{frame}[fragile]
	\frametitle{Contents}
	
	\begin{columns}[t]
		\column{0.5\textwidth}
		\tableofcontents[sections={1-6}]
		
		\column{0.5\textwidth}
		\tableofcontents[sections={7-12}]
	\end{columns}
\end{frame}


\section{Introduction}

\begin{frame}[fragile]
	\frametitle{Introduction}
	
	\begin{itemize}
		\item This course covers fundamental concepts of \textbf{Object-Oriented Programming (OOP)} in Java.
		\item Focus on principles and techniques for \textbf{modular, flexible, and reusable software development}.
		\item Java is a \textbf{object-oriented programming language}, where programs are built using \textbf{objects and classes}.
		\item Understanding OOP principles and software design techniques helps in building \textbf{structured and maintainable applications}.
	\end{itemize}
\end{frame}

\begin{frame}[fragile]
	\frametitle{Core Concepts of OOP in Java}
	
	\begin{itemize}
		\item Every Java program is structured within a \textbf{class}, serving as a blueprint for objects.
		\item A class defines \textbf{attributes} and \textbf{methods} representing an entity’s characteristics and behaviors.
		\item Key OOP principles:
		\begin{itemize}
			\item \textbf{Encapsulation} – Restricts direct access to data.
			\item \textbf{Inheritance} – Allows code reuse by extending existing classes.
			\item \textbf{Polymorphism} – Enables methods to behave differently based on context.
			\item \textbf{Abstraction} – Hides unnecessary details and simplifies system complexity.
		\end{itemize}
	\end{itemize}
\end{frame}

\begin{frame}[fragile]
	\frametitle{Software Design Techniques in Java}
	
	\begin{itemize}
		\item \textbf{Object-Relational Mapping (ORM)} (e.g., Hibernate) allows database interaction without SQL queries.
		\item \textbf{JavaFX} supports event-driven programming and design patterns like:
		\begin{itemize}
			\item \textbf{Model-View-Controller (MVC)}
			\item \textbf{Observer}
			\item \textbf{Dependency Injection}
		\end{itemize}
		\item \textbf{Unified Modeling Language (UML)} diagrams for system design:
		\begin{itemize}
			\item Class Diagram, Object Diagram, Sequence Diagram
			\item Activity Diagram, State Diagram, Use Case Diagram
		\end{itemize}
	\end{itemize}
\end{frame}

\begin{frame}[fragile]
	\frametitle{SOLID Principles in Software Design}
	
	\begin{itemize}
		\item \textbf{Single Responsibility Principle} – Each class should have only one reason to change.
		\item \textbf{Open/Closed Principle} – Classes should be open for extension but closed for modification.
		\item \textbf{Liskov Substitution Principle} – Subclasses should be replaceable without altering expected behavior.
		\item \textbf{Interface Segregation Principle} – Classes should not be forced to implement interfaces they do not use.
		\item \textbf{Dependency Inversion Principle} – High-level modules should not depend on low-level modules; both should rely on abstractions.
	\end{itemize}
\end{frame}

\begin{frame}[fragile]
	\frametitle{Design Patterns in Java}
	
	\begin{itemize}
		\item \textbf{Creational Patterns} – Manage object creation.
		\begin{itemize}
			\item \textbf{Singleton, Factory, Abstract Factory, Builder, Prototype}
		\end{itemize}
		\item \textbf{Structural Patterns} – Define relationships between classes and objects.
		\begin{itemize}
			\item \textbf{Adapter, Decorator, Bridge, Composite, Facade, Proxy, Flyweight}
		\end{itemize}
		\item \textbf{Behavioral Patterns} – Manage interactions between objects.
		\begin{itemize}
			\item \textbf{Strategy, Command, Mediator, Observer, State, Chain of Responsibility, Template Method}
		\end{itemize}
	\end{itemize}
\end{frame}

\begin{frame}
	\frametitle{Learning Outcomes}
	
	\begin{itemize}
		\item Understanding of \textbf{OOP, ORM, JavaFX, UML, SOLID principles, and design patterns}.
		\item Ability to build \textbf{modular, flexible, and reusable applications}.
		\item Foundational knowledge for \textbf{advanced software engineering techniques}.
	\end{itemize}
\end{frame}

\begin{frame}[fragile]
	\frametitle{Course Topics (14 Weeks)}
	
	\begin{columns}[t]
		\column{0.5\textwidth}
		\footnotesize
		\begin{enumerate}
			\item Introduction to Object-Oriented Programming (OOP)
			\item Introduction to ORM (Object-Relational Mapping)
			\item JavaFX and Design Patterns (MVC, Observer, Dependency Injection)
			\item JavaFX Components and Scene Graph
			\item UML Diagrams for Object-Oriented Design (structural modelling)
			\item UML Diagrams for Object-Oriented Design (behavioural modelling)
			\item SOLID Principles in Software Design
		\end{enumerate}
		
		\column{0.5\textwidth}
		\footnotesize
		\begin{enumerate}
			\setcounter{enumi}{7}
			\item Introduction to Design Patterns and Creational Patterns (Part 1)
			\item Advanced Creational Patterns (Part 2)
			\item Basic Structural Patterns (Adapter, Decorator, Bridge)
			\item Advanced Structural Patterns (Composite, Facade, Proxy)
			\item Additional Structural Patterns and Behavioral Patterns (Flyweight, Interpreter, Observer)
			\item Basic Behavioral Patterns (Strategy, Command, Mediator)
			\item Advanced Behavioral Patterns (State, Chain of Responsibility, Template Method)
		\end{enumerate}
	\end{columns}
\end{frame}

\section{Java Program Structure}

\begin{frame}[fragile]
	\frametitle{Java Program Structure}
	
	\begin{itemize}
		\item Java code consists of several key components that form the structure of an application.
		\item Understanding the basic structure is essential for writing well-organized Java programs.
	\end{itemize}
\end{frame}


\begin{frame}[fragile]
	\frametitle{Class}
	
	\begin{itemize}
		\item All Java code must be defined inside a \textbf{class}.
		\item A class acts as a blueprint or template for objects.
	\end{itemize}
	
	\textbf{Example of a class declaration:}
	
	\begin{lstlisting}[style=JavaStyle]
		public class MyClass {
			// Class code here
		}
	\end{lstlisting}
\end{frame}

\begin{frame}[fragile]
	\frametitle{Main Method}
	
	\begin{itemize}
		\item The \textbf{main method} is the entry point of a Java program.
		\item Program execution begins from this method.
	\end{itemize}
	
	\textbf{Syntax of the main method:}
	
	\begin{lstlisting}[style=JavaStyle]
		public static void main(String[] args) {
			// Program code here
		}
	\end{lstlisting}
\end{frame}

\begin{frame}[fragile]
	\frametitle{Attributes and Variables}
	
	\textbf{Attributes:}
	\begin{itemize}
		\item Defined inside a class.
		\item Represent properties or characteristics of an object.
	\end{itemize}
	
	\textbf{Variables:}
	\begin{itemize}
		\item Store temporary data.
		\item Declared inside methods or code blocks.
	\end{itemize}
	
	\textbf{Key differences:}
	\begin{itemize}
		\item \textbf{Attributes} belong to an object and are part of a class or instance.
		\item \textbf{Variables} are used for temporary storage and calculations inside a method.
	\end{itemize}
\end{frame}

\begin{frame}[fragile]
	\frametitle{Attribute Declaration}
	
	\begin{itemize}
		\item Attributes are declared inside a class to represent object characteristics.
	\end{itemize}
	
	\textbf{Example of attribute declaration:}
	
	\begin{lstlisting}[style=JavaStyle]
		class Person {
			String name;
			int age;
		}
	\end{lstlisting}
	
	\begin{itemize}
		\item \texttt{name} and \texttt{age} are attributes of the \texttt{Person} class.
	\end{itemize}
\end{frame}

\begin{frame}[fragile]
	\frametitle{Variable Declaration }
	
	\begin{itemize}
		\item Variables store data and must be declared with a data type before use.
	\end{itemize}
	
	\textbf{Example of variable declaration:}
	
	\begin{lstlisting}[style=JavaStyle]
		int age = 30;
		String name = "John";
	\end{lstlisting}
	
	\begin{itemize}
		\item \texttt{age} and \texttt{name} are variables storing local data inside a method or block.
	\end{itemize}
\end{frame}

\begin{frame}[fragile]
	\frametitle{Methods }
	
	\begin{itemize}
		\item A \textbf{method} is a block of code that performs a specific task.
		\item Methods can be called from other parts of the program.
	\end{itemize}
	
	\textbf{Example of a method:}
	
	\begin{lstlisting}[style=JavaStyle]
		public void greet() {
			System.out.println("Hello!");
		}
	\end{lstlisting}
\end{frame}

\begin{frame}[fragile]
	\frametitle{Comments }
	
	\begin{itemize}
		\item Comments are used to explain code and are not executed.
		\item Java supports two types of comments:
		\begin{itemize}
			\item \texttt{// Single-line comment}
			\item \texttt{/* Multi-line comment */}
		\end{itemize}
	\end{itemize}
	
	\textbf{Example of comments:}
	
	\begin{lstlisting}[style=JavaStyle]
		// This is a single-line comment
		
		/*
		This is a multi-line comment
		*/
	\end{lstlisting}
\end{frame}

\begin{frame}[fragile]
	\frametitle{Import Statement }
	
	\begin{itemize}
		\item The \textbf{import} statement is used to include classes from other packages into a program.
	\end{itemize}
	
	\textbf{Example of an import statement:}
	
	\begin{lstlisting}[style=JavaStyle]
		import java.util.Scanner;
		import all.classes.in.this.package.*;
	\end{lstlisting}
\end{frame}


\begin{frame}[fragile]
	\frametitle{Example Program: \texttt{HelloWorld.java} (Part 1)}
	
	\begin{lstlisting}[style=JavaStyle, caption={Example Program: HelloWorld.java (Part 1)}]
		package hello;
		
		import java.util.Scanner; // Import Scanner class
		
		public class HelloWorld {
			// Main method: Entry point of the program
			public static void main(String[] args) {
				// Declare variable
				String name;
				
				// Create a Scanner object to take user input
				Scanner scanner = new Scanner(System.in);
			\end{lstlisting}
		\end{frame}
		
		\begin{frame}[fragile]
			\frametitle{Example Program: \texttt{HelloWorld.java} (Part 2)}
			
			\begin{lstlisting}[style=JavaStyle, caption={Example Program: HelloWorld.java (Part 2)}]
				// Prompt user for input
				System.out.print("Enter your name: ");
				name = scanner.nextLine();  // Read user input
				
				// Print output with user input
				System.out.println("Hello " + name + "!");
				
				// Close Scanner object
				scanner.close();
			}
		}
	\end{lstlisting}
\end{frame}

\section{Object-oriented Concepts}

\begin{frame}[fragile]
	\frametitle{Java Code: \texttt{MyTest.java}}
	\begin{lstlisting}[style=JavaStyle, caption={Java Code: MyTest.java}]
		package org.pradita.ddp.pertemuan02;
		
		public class MyTest {
			public static void main(String[] args) {
				double a, b;
				a = 3.0;
				b = 4.0;
				double c = Math.sqrt(a * a + b * b);
				System.out.println(c);
			}
		}
	\end{lstlisting}
\end{frame}

\subsection{Abstract Class}
\begin{frame}[fragile]
	\frametitle{Abstract Class (\texttt{Kelas Abstrak})}
	
	\begin{itemize}
		\item An \textbf{abstract class} is a class that \textbf{cannot be instantiated}.
		\item It is commonly used as a \textbf{base class} for other classes.
		\item Abstract classes can have:
		\begin{itemize}
			\item \textbf{Abstract methods} (methods without implementation) that must be implemented by subclasses.
			\item \textbf{Concrete methods} (methods with implementation).
		\end{itemize}
	\end{itemize}
\end{frame}

\begin{frame}[fragile]
	\frametitle{Example of an Abstract Class}
	
	\begin{lstlisting}[style=JavaStyle, caption={Example of an Abstract Class: \texttt{Shape.java}}]
		package edu.example;
		
		public abstract class Shape {
			private String color;
			
			public Shape(String color) {
				this.color = color;
			}
			
			public String getColor() {
				return color;
			}
			
			public abstract double getArea();
		}
	\end{lstlisting}
\end{frame}

\begin{frame}[fragile]
	\frametitle{Explanation of the Abstract Class}
	
	\begin{itemize}
		\item The class \texttt{Shape} is declared as \textbf{abstract}, meaning it cannot be instantiated.
		\item It has:
		\begin{itemize}
			\item An \textbf{abstract method} \texttt{getArea()} that must be implemented by subclasses.
			\item A \textbf{concrete method} \texttt{getColor()} that returns the color of the shape.
		\end{itemize}
		\item Any class that extends \texttt{Shape} must provide an implementation for \texttt{getArea()}.
	\end{itemize}
\end{frame}

\subsection{Inheritance}
\begin{frame}[fragile]
	\frametitle{Inheritance}
	
	\begin{itemize}
		\item \textbf{Inheritance} allows a class (subclass) to inherit attributes and methods from another class (superclass).
		\item Enables \textbf{code reuse} and supports a \textbf{hierarchical class structure}.
		\item A subclass can:
		\begin{itemize}
			\item Access methods and attributes of the superclass.
			\item Add new functionality.
			\item Modify inherited behavior.
		\end{itemize}
	\end{itemize}
\end{frame}

\begin{frame}[fragile]
\frametitle{Example of Inheritance}

\begin{lstlisting}[style=JavaStyle, caption={Example of Inheritance: \texttt{Rectangle.java}}]
package edu.example;

public class Rectangle extends Shape {
	private double width;
	private double height;
	
	public Rectangle(String color, double width, double height) {
		super(color);
		this.width = width;
		this.height = height;
	}
	@Override
	public double getArea() {
		return width * height;
	}
	public double getWidth() {
		return width;
	}
	public double getHeight() {
		return height;
	}
}
\end{lstlisting}
\end{frame}

\begin{frame}[fragile]
	\frametitle{Explanation of Inheritance Example}
	
	\begin{itemize}
		\item The class \texttt{Rectangle} \textbf{extends} the abstract class \texttt{Shape}.
		\item The \texttt{super()} keyword is used to call the constructor of the superclass.
		\item The subclass \texttt{Rectangle} adds:
		\begin{itemize}
			\item Two new attributes: \texttt{width} and \texttt{height}.
			\item A constructor to initialize the color, width, and height.
			\item The implementation of the \texttt{getArea()} method from \texttt{Shape}.
		\end{itemize}
	\end{itemize}
\end{frame}

\subsection{Overriding}
\begin{frame}[fragile]
	\frametitle{Method Overriding (\textbf{Override})}
	
	\begin{itemize}
		\item \textbf{Overriding} allows a subclass to \textbf{replace or modify} a method from its superclass.
		\item The \texttt{@Override} annotation is used to:
		\begin{itemize}
			\item Indicate that a method is intentionally overriding a superclass method.
			\item Prevent errors by ensuring the method signature matches the overridden method.
		\end{itemize}
	\end{itemize}
\end{frame}

\begin{frame}[fragile]
	\frametitle{Example of Method Overriding}
	
	\begin{lstlisting}[style=JavaStyle, caption={Example of Overriding: \texttt{Rectangle.java}}]
		@Override
		public double getArea() {
			return width * height;
		}
	\end{lstlisting}
\end{frame}

\begin{frame}[fragile]
	\frametitle{Explanation of Overriding Example}
	
	\begin{itemize}
		\item The method \texttt{getArea()} in \texttt{Rectangle} \textbf{overrides} the abstract method from \texttt{Shape}.
		\item The \texttt{@Override} annotation ensures that:
		\begin{itemize}
			\item The method correctly replaces the superclass method.
			\item Potential errors (e.g., mismatched method names or parameters) are avoided.
		\end{itemize}
		\item Overriding allows subclasses to provide \textbf{specific implementations} for inherited methods.
	\end{itemize}
\end{frame}

\subsection{Interface}
\begin{frame}[fragile]
	\frametitle{Interface }
	
	\begin{itemize}
		\item An \textbf{interface} is a contract in object-oriented programming that defines methods without providing implementations.
		\item Interfaces allow multiple classes to share common behaviors without requiring direct inheritance.
		\item Any class that implements an interface \textbf{must} provide an implementation for all its methods.
	\end{itemize}
\end{frame}

\begin{frame}[fragile]
	\frametitle{Example of an Interface}
	
	\begin{lstlisting}[style=JavaStyle, caption={Example of an Interface: \texttt{Drawable.java}}]
		package edu.example;
		
		public interface Drawable {
			void draw();
		}
	\end{lstlisting}
\end{frame}

\begin{frame}[fragile]
	\frametitle{Explanation of the Interface Example}
	
	\begin{itemize}
		\item The \texttt{Drawable} interface defines the method \texttt{draw()}.
		\item Any class that implements \texttt{Drawable} must provide its own implementation of \texttt{draw()}.
		\item This allows different classes to implement \texttt{draw()} in a way that suits their behavior.
	\end{itemize}
\end{frame}

\begin{frame}[fragile]
	\frametitle{Implementing an Interface in a Class}
	
	\begin{lstlisting}[style=JavaStyle, caption={Implementing an Interface: \texttt{Rectangle.java}}]
		package edu.example;
		
		public class Rectangle extends Shape implements Drawable {
			private double width;
			private double height;
			
			public Rectangle(String color, double width, double height) {
				super(color);
				this.width = width;
				this.height = height;
			}
			
			@Override
			public double getArea() {
				return width * height;
			}
			
			@Override
			public void draw() {
				System.out.println("Drawing a rectangle with width " + width + " and height " + height);
			}
		}
	\end{lstlisting}
\end{frame}

\begin{frame}[fragile]
	\frametitle{\LARGE{Explanation of \texttt{Rectangle} Class Implementation}}
	
	\begin{itemize}
		\item The class \texttt{Rectangle} extends \texttt{Shape} and implements \texttt{Drawable}.
		\item It overrides:
		\begin{itemize}
			\item \texttt{getArea()} from \texttt{Shape} to calculate the rectangle's area.
			\item \texttt{draw()} from \texttt{Drawable} to define how a rectangle should be drawn.
		\end{itemize}
	\end{itemize}
\end{frame}

\begin{frame}[fragile]
	\frametitle{Implementing an Interface in Another Class}
	
	\begin{lstlisting}[style=JavaStyle, caption={Implementing an Interface: \texttt{Triangle.java}}]
		package edu.example;
		
		public class Triangle extends Shape implements Drawable {
			private double base;
			private double height;
			
			public Triangle(String color, double base, double height) {
				super(color);
				this.base = base;
				this.height = height;
			}
			
			@Override
			public double getArea() {
				return 0.5 * base * height;
			}
			
			@Override
			public void draw() {
				System.out.println("Drawing a triangle with base " + base + " and height " + height);
			}
		}
	\end{lstlisting}
\end{frame}



\begin{frame}[fragile]
	\frametitle{\LARGE{Explanation of \texttt{Triangle} Class Implementation}}
	
	\begin{itemize}
		\item The class \texttt{Triangle} extends \texttt{Shape} and implements \texttt{Drawable}.
		\item It overrides:
		\begin{itemize}
			\item \texttt{getArea()} to calculate the triangle's area.
			\item \texttt{draw()} to define how a triangle should be drawn.
		\end{itemize}
		\item Just like \texttt{Rectangle}, \texttt{Triangle} provides its own unique implementation for both methods.
	\end{itemize}
\end{frame}

\subsection{Polymorphism}
\begin{frame}[fragile]
	\frametitle{Demonstrating Polymorphism with}
	\framesubtitle{\texttt{Rectangle} and \texttt{Triangle}}
	
	\begin{itemize}
		\item \textbf{Polymorphism} allows objects of \texttt{Rectangle} and \texttt{Triangle} to be used uniformly through references to the abstract class \texttt{Shape} or the interface \texttt{Drawable}.
		\item This enables writing flexible and reusable code where different shapes can be processed without needing to know their exact type.
	\end{itemize}
\end{frame}

\begin{frame}[fragile]
	\frametitle{Example: Using Polymorphism in Java}
	
	\begin{lstlisting}[style=JavaStyle, caption={Example of Polymorphism: \texttt{ShapeDemo.java}}]
		package edu.example;
		
		public class ShapeDemo {
			public static void main(String[] args) {
				Shape[] shapes = {
					new Rectangle("blue", 4, 5),
					new Triangle("green", 3, 6)
				};
				
				for (Shape shape : shapes) {
					System.out.println("Color: " + shape.getColor());
					System.out.println("Area: " + shape.getArea());
					
					if (shape instanceof Drawable) {
						((Drawable) shape).draw();
					}
					
					System.out.println();
				}
			}
		}
	\end{lstlisting}
\end{frame}

\begin{frame}[fragile]
	\frametitle{Explanation of the Code}
	
	\begin{itemize}
		\item The array \texttt{shapes} contains objects of \texttt{Rectangle} and \texttt{Triangle}.
		\item Using a reference to \texttt{Shape}, we can call:
		\begin{itemize}
			\item \texttt{getColor()} to retrieve the color of the shape.
			\item \texttt{getArea()} to compute the area without needing to check the shape's type.
		\end{itemize}
		\item The \texttt{instanceof} operator is used to check if an object implements the \texttt{Drawable} interface.
		\item If the object implements \texttt{Drawable}, we cast it to \texttt{Drawable} and call \texttt{draw()}.
	\end{itemize}
\end{frame}

\section{Exception and Unit Test}

\subsection{Unit Test}
\begin{frame}[fragile]
	\frametitle{Unit Testing with JUnit}
	
	\begin{itemize}
		\item \textbf{Unit testing} is a software testing method that verifies small parts of code, such as methods or classes, to ensure they function as expected.
		\item In Java, unit testing is commonly performed using the \textbf{JUnit} framework.
	\end{itemize}
\end{frame}

\begin{frame}[fragile]
	\frametitle{Example of Unit Testing with JUnit}
	
	\begin{lstlisting}[style=JavaStyle, caption={Unit Testing Example: \texttt{RectangleTest.java}}]
		package edu.example.test;
		
		import edu.example.Rectangle;
		import org.junit.jupiter.api.Test;
		import static org.junit.jupiter.api.Assertions.*;
		
		class RectangleTest {
			
			@Test
			void testGetArea() {
				Rectangle rect = new Rectangle("Red", 4, 5);
				assertEquals(20, rect.getArea(), 0.001);
			}
			
			@Test
			void testGetColor() {
				Rectangle rect = new Rectangle("Blue", 4, 5);
				assertEquals("Blue", rect.getColor());
			}
		}
	\end{lstlisting}
\end{frame}

\begin{frame}[fragile]
	\frametitle{Explanation of Unit Testing Example}
	
	\begin{itemize}
		\item \texttt{@Test} is used to declare a test method.
		\item \texttt{assertEquals(expected, actual, delta)} verifies the expected output matches the actual result.
		\item The test ensures that:
		\begin{itemize}
			\item The \texttt{getArea()} method correctly calculates the area.
			\item The \texttt{getColor()} method returns the correct color.
		\end{itemize}
	\end{itemize}
\end{frame}

\subsection{Exception}
\begin{frame}[fragile]
	\frametitle{Exception Handling in Java}
	
	\begin{itemize}
		\item \textbf{Exception handling} is a mechanism in Java for handling errors or unexpected events during program execution.
		\item Exceptions can be handled using the \textbf{\texttt{try-catch}} block.
		\item Custom exceptions can also be created if needed.
	\end{itemize}
\end{frame}

\begin{frame}[fragile]
	\frametitle{Example of Exception Handling}
	
	\begin{lstlisting}[style=JavaStyle, caption={Exception Handling Example: \texttt{ExceptionExample.java}}]
		package edu.example;
		
		public class ExceptionExample {
			public static void main(String[] args) {
				try {
					int result = divide(10, 0);
					System.out.println("Result: " + result);
				} catch (ArithmeticException e) {
					System.out.println("Error: Cannot divide by zero.");
				}
			}
			
			public static int divide(int a, int b) {
				return a / b; // Throws ArithmeticException if b = 0
			}
		}
	\end{lstlisting}
\end{frame}

\begin{frame}[fragile]
	\frametitle{Explanation of Exception Handling Example}
	
	\begin{itemize}
		\item The \textbf{\texttt{try-catch}} block handles the \textbf{\texttt{ArithmeticException}} that occurs when dividing by zero.
		\item If an error occurs:
		\begin{itemize}
			\item The program does not crash.
			\item Instead, the message inside the \texttt{catch} block is displayed.
		\end{itemize}
	\end{itemize}
\end{frame}


\begin{frame}[fragile]
	\frametitle{Creating a Custom Exception in Java}
	
	\begin{itemize}
		\item In Java, custom exceptions can be created by extending the \texttt{Exception} or \texttt{RuntimeException} class.
		\item Custom exceptions allow developers to handle specific error conditions in a structured way.
	\end{itemize}
\end{frame}

\begin{frame}[fragile]
	\frametitle{Example of a Custom Exception}
	
	\begin{lstlisting}[style=JavaStyle, caption={Custom Exception: \texttt{InvalidDimensionException.java}}]
		package edu.example;
		
		class InvalidDimensionException extends Exception {
			public InvalidDimensionException(String message) {
				super(message);
			}
		}
		
		public class CustomExceptionExample {
			public static void main(String[] args) {
				try {
					Rectangle rect = new Rectangle("Red", -5, 10);
				} catch (InvalidDimensionException e) {
					System.out.println("Error: " + e.getMessage());
				}
			}
		}
	\end{lstlisting}
\end{frame}

\begin{frame}[fragile]
	\frametitle{Throwing a Custom Exception in a Class}
	
	\begin{lstlisting}[style=JavaStyle, caption={Throwing a Custom Exception: \texttt{Rectangle.java}}]
		package edu.example;
		
		class Rectangle {
			private double width, height;
			
			public Rectangle(String color, double width, double height) throws InvalidDimensionException {
				if (width <= 0 || height <= 0) {
					throw new InvalidDimensionException("Dimensions must be greater than zero.");
				}
				this.width = width;
				this.height = height;
			}
		}
	\end{lstlisting}
\end{frame}

\begin{frame}[fragile]
	\frametitle{Explanation of Custom Exception Example}
	
	\begin{itemize}
		\item The \texttt{InvalidDimensionException} class extends \texttt{Exception} and defines a constructor that takes an error message.
		\item The \texttt{Rectangle} constructor checks if the width or height is zero or negative.
		\item If invalid values are detected, an \texttt{InvalidDimensionException} is thrown.
		\item The exception is caught in the \texttt{try-catch} block to prevent the program from crashing.
	\end{itemize}
\end{frame}

\begin{frame}[fragile]
	\frametitle{Unit Testing a Custom Exception with JUnit}
	
	\begin{itemize}
		\item JUnit can be used to verify if a method correctly throws an exception.
		\item The \texttt{assertThrows} method ensures that an exception is thrown when invalid input is provided.
	\end{itemize}
\end{frame}

\begin{frame}[fragile]
	\frametitle{Example: Testing Exception with JUnit}
	
	\begin{lstlisting}[style=JavaStyle, caption={Testing Exception with JUnit: \texttt{RectangleExceptionTest.java}}]
		package edu.example.test;
		
		import edu.example.Rectangle;
		import edu.example.InvalidDimensionException;
		import org.junit.jupiter.api.Test;
		import static org.junit.jupiter.api.Assertions.*;
		
		class RectangleExceptionTest {
			
			@Test
			void testInvalidRectangle() {
				Exception exception = assertThrows(InvalidDimensionException.class, () -> {
					new Rectangle("Red", -5, 10);
				});
				assertEquals("Dimensions must be greater than zero.", exception.getMessage());
			}
		}
	\end{lstlisting}
\end{frame}

\begin{frame}[fragile]
	\frametitle{\LARGE{Explanation of JUnit Test for Exception Handling}}
	
	\begin{itemize}
		\item \texttt{assertThrows} is used to check that an \texttt{InvalidDimensionException} is thrown when invalid dimensions are provided.
		\item \texttt{assertEquals} verifies that the error message matches the expected output.
		\item Unit testing with JUnit helps detect issues early by ensuring exceptions are handled correctly.
	\end{itemize}
\end{frame}

\begin{frame}[fragile]
	\frametitle{\LARGE{Importance of Exception Handling and Unit Testing}}
	
	\begin{itemize}
		\item \textbf{Unit Testing:}
		\begin{itemize}
			\item Ensures that code functions as expected.
			\item Automates testing and helps detect bugs early.
		\end{itemize}
		\item \textbf{Exception Handling:}
		\begin{itemize}
			\item Prevents programs from crashing unexpectedly.
			\item Provides meaningful error messages to users and developers.
		\end{itemize}
		\item Proper implementation of these techniques improves:
		\begin{itemize}
			\item Reliability
			\item Scalability
			\item Maintainability
		\end{itemize}
	\end{itemize}
\end{frame}

\end{document}
