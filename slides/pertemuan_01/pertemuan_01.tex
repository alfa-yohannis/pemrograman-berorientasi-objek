\documentclass[aspectratio=169, table]{beamer}

%\usepackage[beamertheme=./praditatheme]{Pradita}
\usepackage[utf8]{inputenc}

\usepackage{listings}

% Define Java language style for listings
\lstdefinestyle{JavaStyle}{
	language=Java,
	basicstyle=\ttfamily\footnotesize,
	keywordstyle=\color{blue},
	commentstyle=\color{gray},
	stringstyle=\color{red},
	breaklines=true,
	showstringspaces=false,
	tabsize=2,
	captionpos=b,
	numbers=left,
	numberstyle=\tiny\color{gray},
	frame=lines,
	backgroundcolor=\color{lightgray!10},
	comment=[l]{//},
	morecomment=[s]{/*}{*/},
	commentstyle=\color{gray}\ttfamily,
	string=[s]{'}{'},
	morestring=[s]{"}{"},
	%	stringstyle=\color{teal}\ttfamily,
	%	showstringspaces=false
}

\lstdefinestyle{sql}{
	language=sql,
	keywords={use, insert, into, values, select, from,
		update, set, delete, create, where, join, left, right, inner, order, by, primary, key},
	ndkeywords={max, min, varchar, int},
	ndkeywordstyle=\color{purple}\bfseries,
	basicstyle=\ttfamily\footnotesize,
	keywordstyle=\color{blue},
	commentstyle=\color{gray},
	stringstyle=\color{red},
	breaklines=true,
	showstringspaces=false,
	tabsize=2,
	captionpos=b,
	numbers=left,
	numberstyle=\tiny\color{gray},
	frame=lines,
	backgroundcolor=\color{lightgray!10},
	comment=[l]{\#},
	morecomment=[s]{/*}{*/},
	commentstyle=\color{gray}\ttfamily,
	string=[s]{'}{'},
	morestring=[s]{"}{"},
	%	stringstyle=\color{teal}\ttfamily,
	%	showstringspaces=false
}


\usetheme{Pradita}

\subtitle{IF220303 - Object-oriented Programming}

\title{\LARGE Foundational Concepts of\\Object-oriented Programming \vspace{10pt}}
\date[Serial]{\scriptsize {PRU/SPMI/FR-BM-18/0222}}
\author[Pradita]{\small {\textbf{Alfa Yohannis}}}

\begin{document}
	
	\frame{\titlepage}
%	
%	
%	{
%		\setbeamertemplate{navigation symbols}{}
%		\setbeamertemplate{footline}{}		
%		\begin{frame}
%			\frametitle{TOGAF Architecture Development Method (ADM)}
%			\framesubtitle{\hspace{1cm}}
%			\vspace{10pt}
%			\begin{center}
%%				\includegraphics[width=0.38\textwidth]{../figures/adm}
%			\end{center}
%		\end{frame}
%	}
	
\begin{frame}[fragile]
	\frametitle{Contents}
	\tableofcontents
\end{frame}


\section{Introduction}

\begin{frame}[fragile]
	\frametitle{Introduction}
	
	\begin{itemize}
		\item This course covers fundamental concepts of \textbf{Object-Oriented Programming (OOP)} in Java.
		\item Focus on principles and techniques for \textbf{modular, flexible, and reusable software development}.
		\item Java is a \textbf{object-oriented programming language}, where programs are built using \textbf{objects and classes}.
		\item Understanding OOP principles and software design techniques helps in building \textbf{structured and maintainable applications}.
	\end{itemize}
\end{frame}

\begin{frame}[fragile]
	\frametitle{Core Concepts of OOP in Java}
	
	\begin{itemize}
		\item Every Java program is structured within a \textbf{class}, serving as a blueprint for objects.
		\item A class defines \textbf{attributes} and \textbf{methods} representing an entity’s characteristics and behaviors.
		\item Key OOP principles:
		\begin{itemize}
			\item \textbf{Encapsulation} – Restricts direct access to data.
			\item \textbf{Inheritance} – Allows code reuse by extending existing classes.
			\item \textbf{Polymorphism} – Enables methods to behave differently based on context.
			\item \textbf{Abstraction} – Hides unnecessary details and simplifies system complexity.
		\end{itemize}
	\end{itemize}
\end{frame}

\begin{frame}[fragile]
	\frametitle{Software Design Techniques in Java}
	
	\begin{itemize}
		\item \textbf{Object-Relational Mapping (ORM)} (e.g., Hibernate) allows database interaction without SQL queries.
		\item \textbf{JavaFX} supports event-driven programming and design patterns like:
		\begin{itemize}
			\item \textbf{Model-View-Controller (MVC)}
			\item \textbf{Observer}
			\item \textbf{Dependency Injection}
		\end{itemize}
		\item \textbf{Unified Modeling Language (UML)} diagrams for system design:
		\begin{itemize}
			\item Class Diagram, Object Diagram, Sequence Diagram
			\item Activity Diagram, State Diagram, Use Case Diagram
		\end{itemize}
	\end{itemize}
\end{frame}

\begin{frame}[fragile]
	\frametitle{SOLID Principles in Software Design}
	
	\begin{itemize}
		\item \textbf{Single Responsibility Principle} – Each class should have only one reason to change.
		\item \textbf{Open/Closed Principle} – Classes should be open for extension but closed for modification.
		\item \textbf{Liskov Substitution Principle} – Subclasses should be replaceable without altering expected behavior.
		\item \textbf{Interface Segregation Principle} – Classes should not be forced to implement interfaces they do not use.
		\item \textbf{Dependency Inversion Principle} – High-level modules should not depend on low-level modules; both should rely on abstractions.
	\end{itemize}
\end{frame}

\begin{frame}[fragile]
	\frametitle{Design Patterns in Java}
	
	\begin{itemize}
		\item \textbf{Creational Patterns} – Manage object creation.
		\begin{itemize}
			\item \textbf{Singleton, Factory, Abstract Factory, Builder, Prototype}
		\end{itemize}
		\item \textbf{Structural Patterns} – Define relationships between classes and objects.
		\begin{itemize}
			\item \textbf{Adapter, Decorator, Bridge, Composite, Facade, Proxy, Flyweight}
		\end{itemize}
		\item \textbf{Behavioral Patterns} – Manage interactions between objects.
		\begin{itemize}
			\item \textbf{Strategy, Command, Mediator, Observer, State, Chain of Responsibility, Template Method}
		\end{itemize}
	\end{itemize}
\end{frame}

\begin{frame}
	\frametitle{Learning Outcomes}
	
	\begin{itemize}
		\item Understanding of \textbf{OOP, ORM, JavaFX, UML, SOLID principles, and design patterns}.
		\item Ability to build \textbf{modular, flexible, and reusable applications}.
		\item Foundational knowledge for \textbf{advanced software engineering techniques}.
	\end{itemize}
\end{frame}

\begin{frame}[fragile]
	\frametitle{Course Topics (14 Weeks)}
	
	\begin{columns}[t]
		\column{0.5\textwidth}
		\footnotesize
		\begin{enumerate}
			\item Introduction to Object-Oriented Programming (OOP)
			\item Introduction to ORM (Object-Relational Mapping)
			\item JavaFX and Design Patterns (MVC, Observer, Dependency Injection)
			\item JavaFX Components and Scene Graph
			\item UML Diagrams for Object-Oriented Design (structural modelling)
			\item UML Diagrams for Object-Oriented Design (behavioural modelling)
			\item SOLID Principles in Software Design
		\end{enumerate}
		
		\column{0.5\textwidth}
		\footnotesize
		\begin{enumerate}
			\setcounter{enumi}{7}
			\item Introduction to Design Patterns and Creational Patterns (Part 1)
			\item Advanced Creational Patterns (Part 2)
			\item Basic Structural Patterns (Adapter, Decorator, Bridge)
			\item Advanced Structural Patterns (Composite, Facade, Proxy)
			\item Additional Structural Patterns and Behavioral Patterns (Flyweight, Interpreter, Observer)
			\item Basic Behavioral Patterns (Strategy, Command, Mediator)
			\item Advanced Behavioral Patterns (State, Chain of Responsibility, Template Method)
		\end{enumerate}
	\end{columns}
\end{frame}

\section{Java Program Structure}

\begin{frame}[fragile]
	\frametitle{Java Program Structure}
	
	\begin{itemize}
		\item Java code consists of several key components that form the structure of an application.
		\item Understanding the basic structure is essential for writing well-organized Java programs.
	\end{itemize}
\end{frame}

\begin{frame}[fragile]
	\frametitle{Class (\textbf{Kelas})}
	
	\begin{itemize}
		\item All Java code must be defined inside a \textbf{class}.
		\item A class acts as a blueprint or template for objects.
	\end{itemize}
	
	\textbf{Example of a class declaration:}
	
	\begin{lstlisting}[style=JavaStyle]
		public class MyClass {
			// Class code here
		}
	\end{lstlisting}
\end{frame}

\begin{frame}[fragile]
	\frametitle{Main Method (\textbf{Metode Utama})}
	
	\begin{itemize}
		\item The \textbf{main method} is the entry point of a Java program.
		\item Program execution begins from this method.
	\end{itemize}
	
	\textbf{Syntax of the main method:}
	
	\begin{lstlisting}[style=JavaStyle]
		public static void main(String[] args) {
			// Program code here
		}
	\end{lstlisting}
\end{frame}

\begin{frame}[fragile]
	\frametitle{Attributes and Variables (\textbf{Atribut dan Variabel})}
	
	\textbf{Attributes:}
	\begin{itemize}
		\item Defined inside a class.
		\item Represent properties or characteristics of an object.
	\end{itemize}
	
	\textbf{Variables:}
	\begin{itemize}
		\item Store temporary data.
		\item Declared inside methods or code blocks.
	\end{itemize}
	
	\textbf{Key differences:}
	\begin{itemize}
		\item \textbf{Attributes} belong to an object and are part of a class or instance.
		\item \textbf{Variables} are used for temporary storage and calculations inside a method.
	\end{itemize}
\end{frame}

\begin{frame}[fragile]
	\frametitle{Attribute Declaration (\textbf{Deklarasi Atribut})}
	
	\begin{itemize}
		\item Attributes are declared inside a class to represent object characteristics.
	\end{itemize}
	
	\textbf{Example of attribute declaration:}
	
	\begin{lstlisting}[style=JavaStyle]
		class Person {
			String name;
			int age;
		}
	\end{lstlisting}
	
	\begin{itemize}
		\item \texttt{name} and \texttt{age} are attributes of the \texttt{Person} class.
	\end{itemize}
\end{frame}

\begin{frame}[fragile]
	\frametitle{Variable Declaration (\textbf{Deklarasi Variabel})}
	
	\begin{itemize}
		\item Variables store data and must be declared with a data type before use.
	\end{itemize}
	
	\textbf{Example of variable declaration:}
	
	\begin{lstlisting}[style=JavaStyle]
		int age = 30;
		String name = "John";
	\end{lstlisting}
	
	\begin{itemize}
		\item \texttt{age} and \texttt{name} are variables storing local data inside a method or block.
	\end{itemize}
\end{frame}

\begin{frame}[fragile]
	\frametitle{Methods (\textbf{Metode})}
	
	\begin{itemize}
		\item A \textbf{method} is a block of code that performs a specific task.
		\item Methods can be called from other parts of the program.
	\end{itemize}
	
	\textbf{Example of a method:}
	
	\begin{lstlisting}[style=JavaStyle]
		public void greet() {
			System.out.println("Hello!");
		}
	\end{lstlisting}
\end{frame}

\begin{frame}[fragile]
	\frametitle{Comments (\textbf{Komentar})}
	
	\begin{itemize}
		\item Comments are used to explain code and are not executed.
		\item Java supports two types of comments:
		\begin{itemize}
			\item \texttt{// Single-line comment}
			\item \texttt{/* Multi-line comment */}
		\end{itemize}
	\end{itemize}
	
	\textbf{Example of comments:}
	
	\begin{lstlisting}[style=JavaStyle]
		// This is a single-line comment
		
		/*
		This is a multi-line comment
		*/
	\end{lstlisting}
\end{frame}

\begin{frame}[fragile]
	\frametitle{Import Statement (\textbf{Import})}
	
	\begin{itemize}
		\item The \textbf{import} statement is used to include classes from other packages into a program.
	\end{itemize}
	
	\textbf{Example of an import statement:}
	
	\begin{lstlisting}[style=JavaStyle]
		import java.util.Scanner;
	\end{lstlisting}
\end{frame}


\begin{frame}[fragile]
	\frametitle{Example Program: \texttt{HelloWorld.java} (Part 1)}
	
	\begin{lstlisting}[style=JavaStyle, caption={Example Program: HelloWorld.java (Part 1)}]
		package hello;
		
		import java.util.Scanner; // Import Scanner class
		
		public class HelloWorld {
			// Main method: Entry point of the program
			public static void main(String[] args) {
				// Declare variable
				String name;
				
				// Create a Scanner object to take user input
				Scanner scanner = new Scanner(System.in);
			\end{lstlisting}
		\end{frame}
		
		\begin{frame}[fragile]
			\frametitle{Example Program: \texttt{HelloWorld.java} (Part 2)}
			
			\begin{lstlisting}[style=JavaStyle, caption={Example Program: HelloWorld.java (Part 2)}]
				// Prompt user for input
				System.out.print("Enter your name: ");
				name = scanner.nextLine();  // Read user input
				
				// Print output with user input
				System.out.println("Hello " + name + "!");
				
				// Close Scanner object
				scanner.close();
			}
		}
	\end{lstlisting}
\end{frame}


\begin{frame}[fragile]
	\frametitle{Java Code: \texttt{MyTest.java}}
	\begin{lstlisting}[style=JavaStyle, caption={Java Code: MyTest.java}]
		package org.pradita.ddp.pertemuan02;
		
		public class MyTest {
			public static void main(String[] args) {
				double a, b;
				a = 3.0;
				b = 4.0;
				double c = Math.sqrt(a * a + b * b);
				System.out.println(c);
			}
		}
	\end{lstlisting}
\end{frame}

\begin{frame}[fragile]
	\frametitle{Abstract Class (\texttt{Kelas Abstrak})}
	
	\begin{itemize}
		\item An \textbf{abstract class} is a class that \textbf{cannot be instantiated}.
		\item It is commonly used as a \textbf{base class} for other classes.
		\item Abstract classes can have:
		\begin{itemize}
			\item \textbf{Abstract methods} (methods without implementation) that must be implemented by subclasses.
			\item \textbf{Concrete methods} (methods with implementation).
		\end{itemize}
	\end{itemize}
\end{frame}

\begin{frame}[fragile]
	\frametitle{Example of an Abstract Class}
	
	\begin{lstlisting}[style=JavaStyle, caption={Example of an Abstract Class: \texttt{Shape.java}}]
		package edu.example;
		
		public abstract class Shape {
			private String color;
			
			public Shape(String color) {
				this.color = color;
			}
			
			public String getColor() {
				return color;
			}
			
			public abstract double getArea();
		}
	\end{lstlisting}
\end{frame}

\begin{frame}[fragile]
	\frametitle{Explanation of the Abstract Class}
	
	\begin{itemize}
		\item The class \texttt{Shape} is declared as \textbf{abstract}, meaning it cannot be instantiated.
		\item It has:
		\begin{itemize}
			\item An \textbf{abstract method} \texttt{getArea()} that must be implemented by subclasses.
			\item A \textbf{concrete method} \texttt{getColor()} that returns the color of the shape.
		\end{itemize}
		\item Any class that extends \texttt{Shape} must provide an implementation for \texttt{getArea()}.
	\end{itemize}
\end{frame}

		
\end{document}
